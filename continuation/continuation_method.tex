\documentclass[11pt,reqno,twoside]{article}

\usepackage{calc}
\usepackage[T1]{fontenc}
\usepackage{times}
\usepackage{lmodern}
%\usepackage{literat}
%\DeclareSymbolFont{operators}{OT1}{\familydefault}{m}{n}
%\addtolength{\textwidth}{-2.2cm} % to emulate the original

\usepackage{amsmath,amssymb,amsfonts}
\usepackage{fullpage}
\usepackage{booktabs}
\usepackage{nicefrac}
\usepackage{siunitx}
\usepackage[numbib]{tocbibind}
\usepackage{wrapfig}
\usepackage[font=small]{caption}
%\usepackage{subcaption}
\usepackage{subfigure}

\usepackage[utf8]{inputenc}
\usepackage{color}
\usepackage{enumitem}
\SetLabelAlign{parright}{\parbox[t]{\labelwidth}{\raggedleft#1}}
\usepackage{psfrag}
\usepackage{amsthm}
\usepackage[english]{babel}
\selectlanguage{english}
%\selectlanguage{catalan}

%\usepackage{eco}
\usepackage[np,autolanguage]{numprint}
  \npdecimalsign{\ensuremath{\cdot}}\npproductsign{\ensuremath{\times}}%
\newcommand{\numold}[1]{\oldstylenums{\numprint{#1}}}

\usepackage{ifpdf}
\usepackage{listings}
\renewcommand{\lstlistingname}{Llistat}

\definecolor{codegreen}{rgb}{0,0.6,0}
\definecolor{codegray}{rgb}{0.5,0.5,0.5}
\definecolor{codepurple}{rgb}{0.58,0,0.82}
\definecolor{backcolour}{rgb}{0.95,0.95,0.92}

%\lstset{language=C, numbers=left, stepnumber=1,
%         basicstyle=\small, numberstyle=\tiny, showspaces=flase}

\lstset{language=matlab,basicstyle=\small}
%, numbers=left, stepnumber=1,
%         basicstyle=\small, numberstyle=\tiny, showspaces=flase}

%\graphicspath{{./figures/}}

%-------------
%pdflatex,
%Macros para producir hyperlinks
\usepackage[pdftex]{graphicx}
\DeclareGraphicsExtensions{.pdf,.jpg,.png,.gif}
\usepackage[pdftex,pagebackref,colorlinks,bookmarksnumbered,
            breaklinks=true,
            colorlinks=true,
            linkcolor=blue,
            citecolor=red,
            urlcolor=magenta]{hyperref}
\hypersetup{
         pdfauthor   = {EquipDocentCalculNumeric},
         pdftitle    = {Template for problems},
         pdfsubject  = {Teaching},
         %pdfpagemode = {FullScreen},
         %pdfstartview = {},
         colorlinks  = {true},
         %bookmarks   = {true},
         %pagebackref = {true},
         bookmarksnumbered = {true},
         %hyperindex  = {true}
}
%\pdfadjustspacing=1
\usepackage{url}

%Disseny de la pàgina
%\usepackage{showframe}
%%%%%%%%%%%%%%%%%%%%%%%%%%%%%%%%%%%%%%%%%%%%%%%%%%%%%%%%%%%%%%%%%%%%%
\usepackage{geometry}
\geometry{
  %papersize={210mm, 296mm},
  a4paper,
  left = 15mm,
  right = 10mm,
  top = 10mm,
  head = 10mm,
  foot = 10mm,
  bottom = 20mm,
  %includeall = false,
}
%%%%%%%%%%%%%%%%%%%%%%%%%%%%%%%%%%%%%%%%%%%%%%%%%%%%%%%%%%%%%%%%%%%%%

\newcommand{\N}{\ensuremath{\mathbb{N}}}
\newcommand{\Z}{\ensuremath{\mathbb{Z}}}
\newcommand{\Q}{\ensuremath{\mathbb{Q}}}
\newcommand{\R}{\ensuremath{\mathbb{R}}}
\newcommand{\C}{\ensuremath{\mathbb{C}}}
\def\ds{\displaystyle}
\def\rme{\mathrm{e}}
\def\rmi{\mathrm{i}}
\def\I{\mathrm{I}}
\def\SVD{\textit{SVD} }
\def\sign{\mathrm{sign}}
\def\diag{\mathrm{diag}}
\def\nuc{\mathrm{Nuc{}}}
\def\rang{\mathrm{Rang{}}}
\def\matlab{
\href{https://es.mathworks.com/products/matlab.html}%
{MATLAB\textsuperscript{\textregistered}}}

\newcommand{\classe}[1]{\ensuremath{\mathcal{C}^{#1}}}
\newcommand{\D}{\ensuremath{\mathcal{D}}}

\newtheorem{thm}{Theorem}[section]
\newtheorem{main_thm}[thm]{Main Theorem}
\newtheorem{cor}[thm]{Corollary}
\newtheorem{lem}[thm]{Lemma}
\newtheorem{prop}[thm]{Proposition}
\newtheorem{defn}[thm]{Definition}
\theoremstyle{remark}
\newtheorem*{rem*}{Remarca}
\newtheorem{rem}{Remarca}%[section]
\newtheorem{nota}{Nota}
\newtheorem{prob}[thm]{Exercici}

\newcommand{\taylor}{\textsf{Taylor} }

\newcommand{\notocsection}[1]{%
    \refstepcounter{section}%
    \section*{\thesection \quad #1}}%

\def\refname{Referències}

\begin{document}
\title{}
\author{}
\date{}
%\maketitle
%\tableofcontents
%\maketitle

\section{Continuation method}\label{sec:pseudoArc} Our goal is to continue
\emph{numerically} a curve $\mathcal{C}\subset\R^{n+1}$, defined
implicitely by the equation
%\begin{displaymath}
       $F(z) = 0$,
%\end{displaymath}
being $F:\R^{n+1}\longrightarrow \R^{n}$ a smooth function. Let us assume
that $z^{j}\in\R^{n+1}$, is a \emph{regular} point of $\mathcal{C}$, so 
%\begin{displaymath}
$F\left(z^{j}\right) = 0$, and
  $\mathop{rank}DF\left(z^{j}\right) = n$.
%\end{displaymath}
Moreover, let $v^{j}\in\R^{n+1}$ be an unitary vector tangent to the curve
$\mathcal{C}$ at the point $z^{j}$, $v^{j}\in T_{z^{j}}\mathcal{C}$, so 
%\begin{displaymath}
  $\| v^{j}\| = 1$, and
  $DF\left(z^{j}\right) v^{j} = 0$.
%    $\mathcal{C}$ at point $z^{j}$). }
%\end{displaymath}

Then, it is possible to find a new point on the curve,
$z^{j+1}\in\mathcal{C}$, and a new unitary tangent vector, to 
$\mathcal{C}$ at $z^{j+1}$, $v^{j+1}\in T_{z^{j+1}}\mathcal{C}$,
$\left\|v^{j+1}\right\| = 1$. If, on its turn, $z^{j+1}$ is a regular point
of $\mathcal{C}$, then  one can look for yet another point on
$\mathcal{C}$, $z^{j+2}\in\mathcal{C}$, and a new unitary tangent vector to
$\mathcal{C}$ at $z^{j+2}$, $v^{j+2}\in T_{z^{j+2}}\mathcal{C}$,
$\left\|v^{j+2}\right\| = 1$, and so on.

Of course, there are several numerical methods to do this
\emph{step-by-step} continuation of $\mathcal{C}$ from an inital point on
the curve, $z^{j}\in\mathcal{C}$, and a (normalizsed) tangent direction at
that point, $v^{j}\in T_{z^{j}}\mathcal{C}$. The one we outline here is the so
called \emph{pseudo-arc continuation method} (see~\cite{Kuznetsov2004},
Chap.~10, Sect.~2, for a complete description). In a nutshell, it consists
in the three stages discussed below.
%\begin{enumerate}[label = \emph{\arabic*.}]
%  \begin{description}  
  \paragraph{Stage $1$: Prediction.}
    Take $\hat{z}^{j+1} = z^{j} + h_{j}
    v^{j}\in z^{j} + \left\langle v^{j}\right\rangle$ as an approximation
    for another new point $z^{j+1}\in\mathcal{C}$. Here $h_{j} > 0$ is the
    pseudo-arc length, and can be conveniently adapted at each step. 
%%
  \paragraph{Stage $2$: Correcton.}
 Refine the approximation $\hat{z}^{j+1}$ to
    find $z^{j+1}\in\R^{n+1}$ such that $F\left(z^{j+1}\right) = 0$.
    However, as the system $F(z) = 0$ has $n$ equations and $n+1$ unknowns
    $z_{1}, z_{2},\dots,z_{n},z_{n+1}$, we need to ask for an additional
    condition: in particular, we shall require that 
    $z^{j+1}\in\hat{z}^{j+1} +\left\langle v^{j}\right\rangle^{\perp}$, i.e., that
    $z^{j+1}$ belongs to
    the hyperplane orthogonal to the vector $v^{j}$ that holds
    $\hat{z}^{j+1}$
    (see Figure~\ref{fig:pseudo-arc}). The  
    corresponding equation con be formulated as
    \begin{align*}
      \left\langle v^{j}, z^{j+1} - \hat{z}^{j+1}\right\rangle &=  
      \left\langle v^{j}, z^{j+1} - z^{j} - h_{j} v^{j}\right\rangle \\
      &= \left\langle v^{j}, z^{j+1}\right\rangle -
        \left\langle v^{j}, z^{j}\right\rangle - 
        h_{j} \left\langle v^{j}, v^{j}\right\rangle\\
      &=  \left\langle v^{j}, z^{j+1}\right\rangle -
          \left\langle v^{j}, z^{j}\right\rangle - h_{j} = 0,
    \end{align*}
    where $\langle\cdot, \cdot\rangle$ stands for the \emph{inner} (or dot)
    product $\left\langle\xi, \eta\right\rangle := \xi_{1}\eta_{1} + \dots +
    \xi_{m}\eta_{m}$, $\xi, \eta\in\R^{m}$. Hence $z^{j+1}$ will be
    given by the solution of the nonlinear system,
    \begin{displaymath}
      \begin{split}%\label{eq:enlarged-system-corrector}
        F(z) &= 0,\\
        \left\langle v^{j}, z\right\rangle &= \left\langle v^{j},
        z^{j}\right\rangle - h_{j}
        \end{split}
    \end{displaymath}
    that can be solved by some iterative method (for example, Newton
    method) taking as initial approximation. $z^{(0)} = \hat{z}^{j+1}$. 
  %  
\paragraph{Stage $3$: Tangent vector.} To find the tangent vector to the
curve at the new point $z^{j+1}\in\mathcal{C}$ found at Stage
$2$, $v^{j+1}\in  T_{z^{j+1}}\mathcal{C}$, first we solve the
$(n+1)$-dimensional linear system
    \begin{equation}\label{eq:appended-system}
      \begin{split}
      DF\left(z^{j+1}\right)v &= 0,\\
      \left\langle v^{j}, v \right\rangle &= 1. 
    \end{split}
  \end{equation}
  As it is pointed out in~\cite{Kuznetsov2004}:
  \begin{enumerate}[label = \emph{(\roman*)}]
    \item If $\mathcal{C}$ is a regular curve and $z^{j}$, $z^{j+1}$ are close
  enough, the system~\eqref{eq:appended-system} is nonsingular.
    \item The solution, $v^{\ast}\in\R^{n+1}$,
  of~\eqref{eq:appended-system} satisfies $\left\langle v^{j},
  v^{\ast}\right\rangle = 1$, so the direction along the curve is
  preserved.
\end{enumerate}
Next, we normalize. If $v^{\ast}\in\R^{n+1}$ denotes the solution
of~\eqref{eq:appended-system}, we divide by its norm, so $v^{j+1} =
v^{\ast}/\left\|v^{\ast}\right\|$. This is the tangent vector we look for. 
%\end{description}

\begin{figure}[!t]
  \centering
  \includegraphics[scale=1.0]{arcstep}
  \caption{We add an extra condition: $z^{j+1}\in
  \hat{z}^{ji+1} + \left\langle v^{j}\right\rangle^{\perp}$.
See~\cite{Kuznetsov2004}, Figure 10.6(b).\label{fig:pseudo-arc}}
\end{figure}

Now, the process can be iterated using the output of Stage $3$, the tangent
vector $v^{j+1}$ to feed Stage $1$ and find a another point close to the
curve  $\mathcal{C}$ and so on. If for the new computed point curve $C$,
and so on. If at Stage $2$, $\mathop{Rank} DF\left(z^{k}\right) < n$, for
the new computed point, $z^{k}\in\mathcal{C}$, then 
 one has to stop the process and analyse for the possible
appearing of branches (bifurcations).

\bibliographystyle{plain}
\bibliography{references}
%\nocite{*}
\end{document}

%%% Local Variables:
%%% mode: latex
%%% TeX-master: "manual"
%%% End:
